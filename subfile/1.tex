%&"../main"
\documentclass[../main]{subfiles}
\begin{document}

\chapter{引言}%
\label{cha:引言}

在当今世界,跨国公司作为多元化经营的典型代表已成为全球经济持续发展的关键。一个工
商企业,在两个或更多的国家经营业务,且该企业有一个反应全球战略经营政策的中央决策
体系,企业内的各个实体分享资源、信息并分担责任,该企业即为狭义上的跨国公司。发达
国家的跨国公司直接投资于发展中国家,通过许可经营的方式为其带去了先进的生产技术、
管理经验和充足的投资资金,有利于缩小全球贫富差距,提高发展中国家发展潜力。

以中国为例,截止去年底,中国前100大跨国公司海外资产总额70862亿元,海外业务收入
47316 亿元,海外员工总人数1011817人,同比增长25.79\%和8.61\% 和34.06\% 中国跨国
公司的数量、质量,随着我国经济的快速发展,发展的规模和实力不断增强。

由此可见,跨国公司的投资与经营已经成为了当前我国外经贸增长方式转变的重要内容之一
。因此,有必要对我国跨国公司进行研究,探讨优化的对策和建议。追求市场价值最大化。

针对研究问题,本文以华为公司为例进行分析,指出其存在的问题,并给出相应的建议。

\end{document}

