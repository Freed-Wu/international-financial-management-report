%&"main"
% Fakesection 检错
% Fakesubsubsection 宏包
\RequirePackage[l2tabu, orthodox]{nag}
% XXX: use nag at first to make all packages effecitive <03-11-19> %
% Fakesubsubsection 编译器
\RequirePackage{ifxetex}
\RequireXeTeX

\documentclass[twoside, openright]{report}
% Fakesection 基础
% Fakesubsection 文字
% Fakesubsubsection 颜色
\usepackage[x11names]{xcolor}
% XXX: conflict with microtype <02-11-19> %
% Fakesubsubsection 长度
\usepackage{printlen}
\uselengthunit{mm}
% Fakesubsubsection 效果
\usepackage{ulem}
% Fakesubsubsection 字体
\usepackage[no-math]{fontspec}
% XXX: use no-math to avoid conflict with mtxlatex <02-11-19> %
\setmainfont{Times New Roman}
% XXX: load ctex before siunitx to avoid \ohm ineffective <04-10-19> %
\usepackage[
	UTF8,
	fontset = windows,
	heading = true,
	zihao = -4,
	sub4section,
]{ctex}
\setCJKfamilyfont{zhsong}[
	AutoFakeBold = 2.17,
	AutoFakeSlant = 0.5,
]{SimSun}
\renewcommand*{\songti}{\CJKfamily{zhsong}}
% XXX: load newtxtext before textcomp to avoid option clash <04-10-19> %
\usepackage{newtxtext}
% Fakesubsubsection 字符边框
\usepackage{varwidth}
% Fakesubsection 断行
\usepackage{fvextra}
% Fakesubsection 标点
% XXX: load csquotes after fvextra to avoid warning <04-10-19> %
\usepackage{csquotes}
% Fakesubsubsection 中文引号
\DeclareQuoteStyle{zh}
{\symbol{"201C}}%“
{\symbol{"201D}}%”
{\symbol{"2018}}%‘
{\symbol{"2019}}%’
\setquotestyle{zh}
% Fakesubsubsection 日文引号
\DeclareQuoteStyle{ja}
{\symbol{"300E}}%『
{\symbol{"300F}}%』
{\symbol{"300C}}%「
{\symbol{"300D}}%」
% Fakesubsubsection 书名号
\DeclareQuoteStyle{book}
{\symbol{"300A}}%《
{\symbol{"300B}}%》
{\symbol{"3008}}%〈
{\symbol{"3009}}%〉
% Fakesubsection 样式
% Fakesubsubsection 目录
\usepackage{titletoc}
\titlecontents{chapter}[0pt]{\filright}{\contentspush{\thecontentslabel}}{}
{\titlerule*{.}\contentspage}
% Fakesubsubsection 章节
\ctexset{
	chapter = {
		name = ,
		number = \arabic{chapter},
		aftername = \hspace{1\ccwd},
		format = \ifthenelse{\value{chapter}=0}{\centering}{}
		\zihao{3}\heiti\bfseries,
		beforeskip = 0.5\ccwd,
		afterskip = 0.5\ccwd,
	},
	section = {
		aftername = \hspace{1\ccwd},
		format = \ifthenelse{\value{chapter}=0}{\centering}{}
		\zihao{-3}\heiti\bfseries,
		beforeskip = 0.5\ccwd,
		afterskip = 0.5\ccwd,
	},
	subsection = {
		format = \zihao{4}\heiti\bfseries,
	}
}
% Fakesubsubsection 图表
\usepackage{caption}
\captionsetup[figure]{labelsep=space}
\captionsetup[table]{labelsep=space}
\captionsetup{font=small}
\DeclareCaptionFont{blue}{\color{LightSteelBlue3}}
\setlength{\abovecaptionskip}{0.5\ccwd}
\setlength{\belowcaptionskip}{0.5\ccwd}
% Fakesubsubsection 子图表
\usepackage{subcaption}
% Fakesubsubsection 公式
\setlength{\abovedisplayskip}{0.5em}
\setlength{\belowdisplayskip}{0.5em}
% Fakesubsubsection 列表
\usepackage{enumitem}
\setlist[enumerate, 1]{
	fullwidth,
	label = (\arabic*),
	font = \textup,
	itemindent = 2em,
}
\setlist[enumerate, 2]{
	fullwidth,
	label = (\alph*),
	font = \textup,
	itemindent = 4em,
}
% Fakesubsubsection 代码
% XXX: need -shell-escape & pygmentize <04-10-19> %
\usepackage{minted}
\usepackage{boxie}
% XXX: conflict with fancybox <02-11-19> %
% Fakesubsubsection 问答
\usepackage{exercise}
\usepackage{tasks}
\renewcommand{\ExerciseName}{问题}
\renewcommand{\AnswerName}{回答}
\renewcommand{\listexercisename}{问题}
% Fakesubsubsection 改动

% Fakesection 插入
% Fakesubsection 表格
% Fakesubsubsection 三线
\usepackage{booktabs}
% Fakesubsubsection 对角线
\usepackage{diagbox}
% Fakesubsubsection 合并列
\usepackage{multicol}
% Fakesubsubsection 合并行
\usepackage{multirow}
% Fakesubsubsection 分割单元格
\usepackage{makecell}
% Fakesubsubsection 短表
\usepackage{tabu}
% Fakesubsubsection 长表
\usepackage{longtable}
% Fakesubsubsection 彩色表
\usepackage{colortbl}
\usepackage{tcolorbox}
\tcbuselibrary{skins}
\tcbuselibrary{breakable}
\tcbuselibrary{theorems}
\tcbuselibrary{listings}
\tcbuselibrary{xparse}
% XXX: need -shell-escape & pygmentize <04-10-19> %
\tcbuselibrary{minted}
% Fakesubsubsection 导入数据
\usepackage{csvsimple}
% Fakesubsection 图形
% Fakesubsubsection 插图
\usepackage{graphicx}
\graphicspath{{fig/}{fig/\arabic{chapter}/}}
% XXX: use '../' to make subfiles can find path <03-11-19> %
% Fakesubsubsection 环绕
\usepackage{wrapfig}
% Fakesubsubsection 图片重叠
\usepackage{overpic}
% Fakesubsubsection 徽标
\usepackage{hologo}
% Fakesubsubsection 条形码
\usepackage{ean13isbn}
% Fakesubsubsection 二维码
\usepackage{qrcode}
% Fakesubsection 符号
% Fakesubsubsection 数学符号
\usepackage{newtxmath}
\usepackage{bm}
% Fakesubsubsection 幻灯片符号
\usepackage{pifont}
% Fakesubsubsection 数学符号放缩
\usepackage{relsize}
% Fakesubsubsection 公式
\usepackage{cases}
\usepackage{physics}
% Fakesubsubsection 单位
\usepackage{siunitx}
\sisetup{mode=text}
% Fakesubsubsection 计算机
% Fakesubsection 媒体
\usepackage{media9}
% Fakesubsection 链接
\usepackage{hyperref}
\hypersetup{
	colorlinks = true,
	linkcolor = gray!50!black,
	citecolor = gray!50!black,
	backref = page
}
% Fakesubsection 批注
\usepackage{todonotes}
\usepackage{cooltooltips}
\usepackage{pdfcomment}
% Fakesubsection 文本框
\usepackage{boxedminipage2e}
% Fakesubsection 页眉页脚
\usepackage{fancyhdr}
\fancypagestyle{plain}{
	\pagestyle{fancy}
}

% Fakesection 设计
% Fakesubsection 水印
\usepackage{wallpaper}
% Fakesubsection 音乐
\usepackage{mtxlatex}
\mtxlatex
% Fakesubsection 主题

% Fakesection 布局
% Fakesubsection 页面
\usepackage{geometry}
% Fakesubsection 缩进
\usepackage{indentfirst}
% Fakesubsection 间距
\usepackage{setspace}
\usepackage[
	restoremathleading=false,
	UseMSWordMultipleLineSpacing,
	MSWordLineSpacingMultiple=1.5
]{zhlineskip}
% Fakesection 引用
% Fakesubsection 脚注
\renewcommand{\thefootnote}{\fnsymbol{footnote}}
\renewcommand{\thempfootnote}{\fnsymbol{mpfootnote}}
% Fakesubsection 引文
\usepackage{morewrites}
\usepackage[
	square, comma, numbers, super, sort&compress, longnamesfirst, sectionbib,
	nonamebreak
]{natbib}
% Fakesubsection 题注
\usepackage{epigraph}
% Fakesubsection 索引
\usepackage{makeidx}
\makeindex
% Fakesubsection 关联
% XXX: need amsmath <04-10-19> %
\numberwithin{Exercise}{chapter}
\numberwithin{Answer}{chapter}

% Fakesection 特殊功能
% Fakesubsection 页数统计
\usepackage{lastpage}
% Fakesubsection 数学表达式
\usepackage{calc}
% Fakesubsection 条件编译
\usepackage{subfiles}
% XXX: use subfiles at last to make all packages effecitive <03-11-19> %

\begin{document}

% Fakesection 扉页

\newcommand{\Title}{以华为为例分析跨国公司融资结构}

\begin{titlepage}
	\centering
	\makebox[4\ccwd][s]{}

	\textbf{\fontsize{25pt}{\baselineskip}\kaishu{南京理工大学经济管理学院}}

	\vspace{10mm}

	\textbf{\fontsize{33pt}{\baselineskip}\lishu{课程论文}}

	\vspace{20mm}

	\begin{table}[htpb]
		\centering
		\songti
		\zihao{3}
		\begin{tabu}to.8\linewidth{@{}X[4,r]@{}X[c]@{}X[18,l]@{}}
			\makebox[4\ccwd][s]{课程名称} & :                                             &
			\underline{\makebox[18\ccwd][c]{国际财务管理}}                                   \\
			\makebox[4\ccwd][s]{论文题目} & :
			                              & \underline{\makebox[18\ccwd][c]{\Title}}         \\
			\makebox[4\ccwd][s]{姓名}     & :
			                              & \underline{\makebox[18\ccwd][c]{吴振宇}}         \\
			\makebox[4\ccwd][s]{学号}     & :
			                              & \underline{\makebox[18\ccwd][c]{916101630117}}   \\
			\makebox[4\ccwd][s]{成绩}     & :
			                              & \underline{\makebox[18\ccwd][c]{}}               \\
			\makebox[4\ccwd][s]{时间}     & :
			                              & \underline{\makebox[18\ccwd][c]{\today}}
		\end{tabu}
	\end{table}

	\begin{boxedminipage}{\linewidth}
		\zihao{4}

		任课老师评语:

		\vspace{40mm}

		\begin{flushright}
			签名: \underline{\makebox[6\ccwd][c]{}}

			年\makebox[2\ccwd][c]{}月\makebox[2\ccwd][c]{}日
		\end{flushright}

		\vspace{20mm}

	\end{boxedminipage}

\end{titlepage}

% Fakesection 摘要页眉页脚

\pagestyle{fancy}
\renewcommand{\headrulewidth}{0pt}
\fancyhead[LC, RC]{}
\fancyhead[LE, RO]{}
\fancyhead[RE, LO]{}
\fancyfoot[LC, RC]{}
\fancyfoot[LE, RO]{}
\fancyfoot[RE, LO]{}

% Fakesection 摘要

\newpage

\begin{center}

	\vspace{20pt}

	\heiti\zihao{2}\textbf{\Title}

	\vspace{20pt}

	\kaishu\zihao{4}\textbf{916101630117~吴振宇}

\end{center}

\vspace{2em}

{

	\setlength{\baselineskip}{18pt}
	\kaishu

	\zihao{5}
	\noindent

	\textbf{摘要:}\hspace{\ccwd}随着我国跨国公司的不断发展壮大,跨国公司数
	量和质量的不断提高,一些跨国公司存在的共性问题也预演愈烈。为了寻求解决这
	些问题的方法与对策,并针对跨国公司寻求利益最大化的目标,本文以华为公司为
	例,从外汇、融资等方面分析了该公司的问题,并给出了最终的建议。
	\cite{王建英2003国际财务管理学,宋常2001国际财务管理的整体构架,王化成1996国际财务管理目标研究}

	\textbf{关键词}\hspace{\ccwd}外汇管理\hspace{\ccwd}融资结构

	\zihao{5}
	\noindent

	\textbf{Abstract:}\hspace{1em}With the continuous development and growth
	of China's multinational companies, the number and quality of
	multinational companies continue to improve, and some common problems of
	multinational companies are also increasingly rehearsed. In order to
	find solutions to these problems, and to maximize the benefits of
	multinational companies, this paper takes Huawei as an example, analyzes
	the problems of Huawei from foreign exchange, financing and other
	aspects, and gives the final recommendations.

	\textbf{Key Words:}\hspace{1em}Exchange Control\hspace{1em}Financing
	Structure

}

% Fakesection 目录页眉页脚

\newpage

\renewcommand{\headrulewidth}{0.4pt}
\fancyhead[LC, RC]{\small\leftmark}
\fancyhead[LE, RO]{第\thepage 页~共~\pageref{LastPage}~页}
\fancyhead[RE, LO]{\small\rightmark}

% Fakesection 目录

\pagenumbering{roman}

\setcounter{tocdepth}{2}

\tableofcontents
\listoffigures
\listoftables

\pagenumbering{arabic}

% Fakesection 正文

\subfile{subfile/1}
\subfile{subfile/2}
\subfile{subfile/3}
\subfile{subfile/4}

% Fakesection 参考文献

\setcounter{chapter}{0}

\addcontentsline{toc}{chapter}{参考文献}

\bibliographystyle{gbt7714-plain}
\bibliography{bib/main}

\end{document}

